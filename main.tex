\documentclass[11pt]{article}
\usepackage[utf8]{inputenc}
\usepackage[T1]{fontenc}
\usepackage{hyperref}
\usepackage[authoryear]{natbib}
\usepackage{textcomp}
\usepackage{subcaption}
\usepackage{graphicx}
\usepackage{fancybox}
\usepackage{xcolor}
\usepackage{makeidx}
\usepackage{amsmath,amssymb}
\usepackage{eurosym}
\usepackage[toc,page]{appendix}
\usepackage{array,multirow,makecell}
\renewcommand{\arraystretch}{0.5}
\setcellgapes{1pt}
\makegapedcells
\renewcommand*\thetable{\Roman{table}}
\newcolumntype{R}[1]{>{\raggedleft\arraybackslash }b{#1}}
\newcolumntype{L}[1]{>{\raggedright\arraybackslash }b{#1}}
\newcolumntype{C}[1]{>{\centering\arraybackslash }b{#1}}
\usepackage[left=2.2cm,right=2.2cm,top=2.5cm,bottom=2.5cm]{geometry}
\linespread{1.6}
\date{}
% Suggestion de titre :
\setlength{\parindent}{0pt}\title{\textsc{Brouillon} - French Preferences over Climate Policies}
\author{Thomas Douenne and Adrien Fabre\footnote{\scriptsize Douenne: Paris School of Economics, Université Paris 1 Panthéon-Sorbonne, 48 Boulevard Jourdan, 75014, Paris, France (email: thomas.douenne@psemail.eu); Fabre: Paris School of Economics, Université Paris 1 Panthéon-Sorbonne, 48 Boulevard Jourdan, 75014, Paris, France (email: adrien.fabre@ens.fr)}}
\date{March 2019}

\begin{document}

\maketitle

\vspace*{1.3cm}

{\Large \textbf{Disclaimer: This is a work in progress and, therefore, a draft version of the final paper.
Please, do not cite without the authors' permission.}}

\vspace*{2em}\begin{center}
\textbf{Abstract:} 
\end{center}

\hfill \break
\hspace*{1cm} \parbox{15cm}{\noindent {\small \textit{...}}}

\vspace*{2cm}


JEL classification: ...
% Suggestions : ?

Keywords: Climate Policy; Carbon tax; Preferences; France; Perceptions 
% Suggestions : ?

\newpage
\tableofcontents

\newpage
\section{Introduction}

% Idée de plan ? On peut parler (dans le désordre) 1) des déterminants de l'acceptation / 2) perceptions des bénéfices et coûts de la taxe / 3) préférences relatives entre mesures et entre modes de redistribution / 4) la spécificité des gilets jaunes vis-à-vis de ces politiques, et leur caractérisation socio-démographique / 5) d'une éventuelle différence entre "gagnant" et "gagnant pouvoir d'achat" / 6) des préférences vis-à-vis des autres politiques ...
% Il faut aussi présenter les données, je ne sais pas dans quelle mesure on peut reprendre ce qui est déjà fait dans l'autre papier en développant davantage les questions qui nous intéressent. ===> on fait l'inverse, i.e. les questions avec traitement sur la taxe carbone sont décrites dans le papier compagnon, ici nous nous attardons sur les autres questions ---> Ok mais il y a des parties communes, comme sur la méthode d'échantillonage, la représentativité, etc. ===> oui, il faut faire doublon dans ce cas je crois

% Il faut trouver une histoire qui permette d'articuler ces éléments de manière logique et problématisée, si possible essayer d'aller au-delà d'une simple description des résultats du sondage.
% Là aussi il faudrait identifier un journal pour savoir comment organiser l'article. Trois suggestions : un journal scientifique (type Nature Climate Change), une revue d'économie généraliste en français (type Revue économique) ou non, un journal en économie de l'environnement (lequel ?). ===> je crois pas que Nature CC présente des résultats spécifiques, si ? c'est plutôt des synthèses de la littérature je crois. En éco de l'environnement pourquoi pas, mais je viserais plutôt une revue française (on s'emmerdera moins pour réussir à le faire publier), ce qui n'empêche pas d'écrire le papier en anglais. ---> Ok

% Annales d'Économie et Statistiques ne publie que des articles en anglais
%la revue française d'économie n'accepte pas les soumissions en anglais
%la revue d'économie politique si, mais c'est vraiment pas courant

%Papers submitted to the Annals of Economics and Statistics should contain original work that has not been published elsewhere. Authors should not submit their manuscript (or a translation of the text) to another journal before receiving the final decision of the Editors. Submitted articles should be in English, in a clear, concise and direct style. Articles should be submitted in pdf format, and should include an abstract of not more than 150 words, keywords and JEL codes.
%Submitted manuscripts should be formatted for standard-sized paper with margins of at least 1.25 inches on all sides, 1.5 or double-spaced text, and 12 point font. We strongly encourage authors to submit manuscripts that are under or about 45 pages / 17,000 words, all included. The first page of the manuscript should contain the title of the article, and the surnames and first names of all the authors, together with their postal and e-mail addresses, telephone and fax numbers. The corresponding author should be identified.


%%%%%%%%%%%%%%%%%%%%%%%%%%%%%%%%%%%%%%%%%%%%%%%%ù
% Proposition alternative
% L'idée est de partir du CC pour montrer que l'objectif est relativement compris/partagé, expliquer le rejet de la taxe, notamment à cause de problèmes intrinsèques à cette mesure (les problèmes liés au répondant étant discutés en détails dans le 1er papier), voir si d'autres mesures constitueraient de bonnes alternatives aux yeux des gens, et enfin voir si on a eu raison de parler de la majorité comme d'un tout dans les 3 premières parties (R^2 faible des déterminants) ou s'il y a vraiment des catégories avec intérêts/perceptions divergents.
% Pour résumer c'est: pourquoi sont-ils contre tax and dividend (1 pas par négligence du climat mais 2 à cause des problèmes perçus de cette mesure)? pour quoi sont-ils donc, eux qui se disent prêts à changer? (tout ce qui ne leur paraît pas être un changement personnel contraint)

\section{Survey and data}

    \subsection{Data collection}

% TODO: present the survey, questions on socio-demo, its representativeness, etc.

    \subsection{Eliciting attitudes towards climate change and climate policies}
    
% TODO: present more specifically the question that interest us for this paper.



\section{Attitudes over Climate Change}

% Idée : les gens ne connaissent pas la science mais sentent bien que c'est la merde. Puis discuter du niveau de priorité qu'ils y accordent TODO: Regarder d'autres sources pour voir s'ils sont fatalistes/désabusés ("c'est déjà foutu" ou "ça ne changera pas") ou s'ils croient qu'on peut s'en sortir, et la priorité accordée à l'écologie
% autres études: http://www.datapressepremium.com/rmdiff/2008572/Etude-OpinionWay-pour-PrimesEnergie.fr.pdf
% https://www.lejdd.fr/Politique/sondage-tva-redevance-audiovisuelle-les-francais-plebiscitent-les-suppressions-de-taxes-3887510
% https://www.lemonde.fr/idees/article/2019/04/25/la-cohesion-de-la-societe-francaise-n-est-pas-aussi-menacee-qu-on-le-croit_5454574_3232.html
% http://closup.umich.edu/national-surveys-on-energy-and-environment/
% USA: http://environment.yale.edu/climate-communication-OFF/files/Climate-Policy-Report-April-2013-Revised.pdf https://cces.gov.harvard.edu/ https://news.gallup.com/poll/1615/environment.aspx
% world: https://climatecommunication.yale.edu/wp-content/uploads/2016/02/2009_07_International-Public-Opinion.pdf

    \subsection{Knowledge}

% faible connaissance
% Questions à exploiter : causes / GES - Climate Call / région / générations / réductions nécessaires (curseur)


    \subsection{Opinions}

% inquiétude répandue. Qui est tenu responsable.
% Questions à exploiter : fréquence parle / conséquences / responsables / écologiste (?)

    \subsection{The Reaction Needed}

% réalisent-ils l'ampleur de la transition à opérer ? Sont-ils prêts à contribuer ?
% Questions à exploiter : décisions fertilité / prêt à changer / mode de vie écolo


\section{Attitudes over Carbon Tax and Dividend}

% Les économistes ont montré que la taxe est le mécanisme le plus efficace, or les gens la rejettent. Pourquoi ? Donner les raisons et discuter s'ils ont tort, sont égoïstes ou s'ils ont de bonnes raisons de s'y opposer

    \subsection{Massive rejection}

% comparer nos résultats avec d'autres enquêtes et discuter le contexte des gilets jaunes dans l'explication d'un tel rejet
% dire que les gens sont biaisés et renvoyer vers l'autre papier. e.g. how many think they lose from housing although they don't use hydrocarbons? 30%, 90% think to gain less than 50

    \subsection{Perceived benefits}

% inclure aussi la question sur efficacité environnementale, et dire que la taxe envisagée réduirait les émissions de façon marginale

    \subsection{Perceived problems}

% en profiter pour expliquer que la réforme suédoise ("la réforme du siècle") a fonctionné car elle s'inscrivait dans une refondation global du système fiscal, qui traitait en même temps les inégalités (un truc qu'on ne teste pas dans notre enquête mais qui a peut-être été fait ailleurs, c'est le sentiment "c'est encore les pauvres qui vont devoir changer, les riches auront toujours les moyens de polluer autant). Transition avec manque de cohérence et taxe sur le kérosène.

% Murray & Rivers (2015) showed increased support overtime in British Columbia (reject before implementation, support afterwards)

    \subsection{Perceived elasticities}

% Besides, 54\% (resp. 61\%) of respondents report to be inelastic for transport (resp. housing): this is mainly due to mobility constraints for transport (64\% of cases) while it mostly reflects a non-fossil type of heating for housing (61\%).


\section{Attitudes over Other Policies}

    \subsection{Other Instruments}

% (inclure autres sources) est-ce que pour autant les gens soutiendraient d'autres réformes qui permettraient de lutter contre le CC ? ...
% Questions à exploiter : taxation du disel / gaz de schiste / matrice des différentes politiques

    \subsection{Preferred Revenue Recycling}

% ... pas totalement. Alors est-ce qu'ils accepteraient la taxe si elle était redistribuée autrement ? Difficile de savoir en absolu (cf. sondage les échos), mais on peut donner les préférences relatives.

    \subsection{Mobility and Rural Areas}

% résultats sur les habitudes de transport et le souhait de davantage d'alternatives à la voiture, problématique urbanistique/sociale de la désertion des bourgs et des milieux ruraux
%" Dans cette optique, investissements publics verts et taxe carbone apparaissent bien complémentaires, et dans le timing de la politique climatique il serait justifié de réaliser les premiers avant de mettre en place la seconde" note CAE
% "La tendance structurelle à l’étalement urbain doit être infléchie pour réduire les émissions du secteur des transports, ce qui requiert des villes accessibles financièrement et attrayantes." note CAE

\section{Determinants of Attitudes}

% Discuter à quels points ces déterminants expliquent les attitudes, vs. la part d'idiosyncrasie dans les réponses. Et s'intéresser aux déterminants de toutes les questions, pas seulement ceux sur l'acceptation de tax and dividend.

% TODO: stat descriptive of acceptance rate among those who have correct beliefs, as well as who they are and acceptance rate of similar persons. Matching with similar ones who have incorrect beliefs ? Or rather, average estimated acceptance given regression on socio-demo.

    \subsection{Attitudes over climate change}

    \subsection{Attitudes over policies}

% ANOVA
% McFadden pseudo-R^2, as in Kalbekken & Saelen 

\begin{table}[!htbp] \centering 
  \caption{Determinants of attitudes towards climate change} 
  \label{tab:bias} 
\makebox[\textwidth][c]{ \begin{tabular}{@{\extracolsep{5pt}}lcccccc} 
\\[-1.8ex]\hline 
\hline \\[-1.8ex] 
 & \multicolumn{6}{c}{\textit{Dependent variable:}} \\ 
\cline{2-7} 
\\[-1.8ex] & \multicolumn{2}{c}{CC anthropic} & \multicolumn{2}{c}{CC disastrous} & \multicolumn{2}{c}{Knowledge on CC} \\ 
\\[-1.8ex] & \textit{OLS} & \textit{normal} & \textit{OLS} & \textit{normal} & \textit{OLS} & \textit{normal} \\ 
\hline \\[-1.8ex] 
 Constant & 0.274$^{*}$ &  & 0.305$^{*}$ &  & 0.480$^{***}$ &  \\ 
  & (0.163) &  & (0.179) &  & (0.071) &  \\ 
  Gilets\_jaunescomprend & $-$0.040$^{*}$ & $-$0.036 & $-$0.051$^{**}$ & $-$0.041$^{*}$ & $-$0.015 & $-$0.013 \\ 
  & (0.022) & (0.023) & (0.024) & (0.024) & (0.010) & (0.025) \\ 
  Gilets\_jaunessoutient & $-$0.106$^{***}$ & $-$0.107$^{***}$ & $-$0.064$^{**}$ & $-$0.055$^{**}$ & $-$0.026$^{**}$ & $-$0.023 \\ 
  & (0.024) & (0.025) & (0.026) & (0.025) & (0.010) & (0.027) \\ 
  Gilets\_jaunesest\_dedans & $-$0.216$^{***}$ & $-$0.201$^{***}$ & $-$0.105$^{**}$ & $-$0.081$^{*}$ & $-$0.047$^{**}$ & $-$0.048 \\ 
  & (0.043) & (0.048) & (0.047) & (0.044) & (0.019) & (0.049) \\ 
  sexeMasculin & $-$0.010 & $-$0.010 & $-$0.005 & 0.002 & 0.071$^{***}$ & 0.074$^{***}$ \\ 
  & (0.019) & (0.018) & (0.020) & (0.020) & (0.008) & (0.021) \\ 
  diplome4 & 0.030$^{***}$ & 0.033$^{***}$ & 0.028$^{***}$ & 0.029$^{***}$ & 0.014$^{***}$ & 0.015 \\ 
  & (0.009) & (0.009) & (0.010) & (0.010) & (0.004) & (0.011) \\ 
  taille\_menage & $-$0.048 & $-$0.274 & $-$0.020 & $-$0.032 & 0.002 & 0.003 \\ 
  & (0.036) & (0.185) & (0.040) & (0.041) & (0.016) & (0.041) \\ 
  taille\_agglo & 0.018$^{**}$ & 0.018$^{**}$ & 0.015$^{*}$ & 0.017$^{**}$ & 0.004 & 0.004 \\ 
  & (0.007) & (0.007) & (0.008) & (0.008) & (0.003) & (0.008) \\ 
  age\_18\_24 & 0.079 & 0.063 & 0.081 & 0.076 & 0.001 & $-$0.005 \\ 
  & (0.054) & (0.048) & (0.060) & (0.060) & (0.024) & (0.061) \\ 
  age\_25\_34 & 0.168$^{***}$ & 0.134$^{***}$ & 0.132$^{***}$ & 0.128$^{***}$ & $-$0.015 & $-$0.017 \\ 
  & (0.044) & (0.034) & (0.048) & (0.048) & (0.019) & (0.051) \\ 
  age\_35\_49 & 0.100$^{**}$ & 0.074$^{**}$ & 0.136$^{***}$ & 0.123$^{***}$ & $-$0.011 & $-$0.015 \\ 
  & (0.041) & (0.036) & (0.045) & (0.043) & (0.018) & (0.045) \\ 
  age\_50\_64 & 0.111$^{***}$ & 0.087$^{***}$ & 0.061$^{*}$ & 0.052$^{*}$ & $-$0.014 & $-$0.012 \\ 
  & (0.029) & (0.024) & (0.032) & (0.030) & (0.013) & (0.032) \\ 
  humaniste & 0.073$^{***}$ & 0.063$^{**}$ & 0.055$^{*}$ & 0.066$^{**}$ & 0.037$^{***}$ & 0.037 \\ 
  & (0.028) & (0.027) & (0.031) & (0.031) & (0.012) & (0.032) \\ 
  ecologiste & 0.146$^{***}$ & 0.153$^{***}$ & 0.191$^{***}$ & 0.182$^{***}$ & 0.047$^{***}$ & 0.045 \\ 
  & (0.026) & (0.022) & (0.028) & (0.028) & (0.011) & (0.028) \\ 
  Gauche\_droiteLeft & $-$0.027 & $-$0.042 & 0.058 & 0.060 & 0.002 & 0.011 \\ 
  & (0.057) & (0.069) & (0.063) & (0.064) & (0.025) & (0.065) \\ 
  Gauche\_droiteCenter & $-$0.086 & $-$0.127$^{*}$ & $-$0.026 & $-$0.020 & 0.010 & 0.022 \\ 
  & (0.060) & (0.076) & (0.065) & (0.065) & (0.026) & (0.068) \\ 
  Gauche\_droiteRight & $-$0.121$^{**}$ & $-$0.166$^{**}$ & $-$0.064 & $-$0.072 & $-$0.005 & 0.006 \\ 
  & (0.059) & (0.075) & (0.064) & (0.064) & (0.026) & (0.067) \\ 
  Gauche\_droiteExtreme-right & $-$0.111$^{*}$ & $-$0.153$^{*}$ & $-$0.037 & $-$0.041 & $-$0.050$^{*}$ & $-$0.039 \\ 
  & (0.061) & (0.079) & (0.067) & (0.067) & (0.027) & (0.071) \\ 
  Gauche\_droiteIndeterminate & $-$0.102$^{*}$ & $-$0.133$^{**}$ & $-$0.042 & $-$0.044 & $-$0.018 & $-$0.008 \\ 
  & (0.058) & (0.067) & (0.063) & (0.063) & (0.025) & (0.066) \\ 
 \hline \\[-1.8ex] 
Observations & 2,746 & 2,746 & 2,746 & 2,746 & 2,746 & 2,746 \\ 
R$^{2}$ & 0.098 &  & 0.119 &  & 0.183 &  \\ 
\hline 
\hline \\[-1.8ex] 
\textit{Note:}  & \multicolumn{6}{r}{$^{*}$p$<$0.1; $^{**}$p$<$0.05; $^{***}$p$<$0.01} \\ 
\end{tabular} 
} \end{table} 


\end{document}
